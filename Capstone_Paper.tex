% template for papers with a title page
% see dgstpp.sty for title page info
% format: latex
% last changed: 1 Apr 2015

\documentclass[11 pt]{article}

% standard math packages
\usepackage{amsmath,amsfonts,amssymb}

% Phil Parker's DGS packages, some modified
\usepackage{remexpp,pprroof,dgstpp}

% other packages
\usepackage{setspace, esvect}
%\usepackage{hyperref,color}

% "fancy" font
\usepackage{fourier}
\usepackage[T1]{fontenc}

% make reference header the right font size
\renewcommand\refname{\Large References}

% theorems, remarks, etc using Phil Parker's "remexpp.sty"
\newtheorem{theorem}{Theorem}[section]
\newtheorem{prop}[theorem]{Proposition}
\newtheorem{lemma}[theorem]{Lemma}
\newtheorem{claim}[theorem]{Claim}
\newtheorem{corollary}[theorem]{Corollary}
\newremark{definition}[theorem]{Definition}
\newremark{example}[theorem]{Example}
\newremark{remark}[theorem]{Remark}

% make rsfs, TeX \cal, and Euler script *all* available
\usepackage{mathrsfs}
\let\rscr=\mathscr
\let\mathscr=\relax
\let\mcal=\mathcal
\usepackage{eucal}
\let\escr=\mathcal
\let\mathcal=\relax

% commutative diagrams with XY-pic
\usepackage[all]{xy}
\SelectTips{cm}{}

\arraycolsep .2em

% new commands
\renewcommand{\a}{\alpha}
\newcommand{\Aut}[1]{\textrm{Aut}(#1)}
\newcommand{\B}{\rscr{B}}
\newcommand{\br}[2]{\left[#1,#2\right]}
\newcommand{\bre}{\br{\ }{\,}}
\newcommand{\ddg}{\ddot{\g}}
\newcommand{\dg}{\dot{\g}}
\newcommand{\DGS}{D{\kern-.375em}G{\kern-.2em}S}
\newcommand{\ds}{\oplus}
\newcommand{\eB}{\escr{B}}
\newcommand{\eH}{\escr{H}}
\newcommand{\eI}{\escr{I}}
\newcommand{\eV}{\escr{V}}
\newcommand{\g}{\gamma}
\newcommand{\G}{\Gamma}
\newcommand{\h}{\lal{h}}
\renewcommand{\H}{\rscr{H}}
\newcommand{\hp}{\h_{2p + 1}}
\newcommand{\iso}{\cong}
\newcommand{\lag}{\mathfrak{g}}
\newcommand{\lal}[1]{\mathfrak{#1}}
\newcommand{\n}{\lal{n}}
\newcommand{\pplus}{+\mspace{-10 mu}+}
\newcommand{\R}{\mathbb{R}}
\newcommand{\rS}{\rscr{S}}
%\renewcommand{\span}[1]{[\mspace{-3.25 mu}[ #1 ]\mspace{-3.25 mu}]}
\newcommand{\surj}{\rightarrow\kern-.82em\rightarrow}
\newcommand{\tQ}{\widetilde{Q}}
\renewcommand{\v}{\lal{v}}
\newcommand{\V}{\rscr{V}}
\newcommand{\z}{\lal{z}}
\newcommand{\fg}{\mathfrak{g}}
\newcommand{\fz}{\mathfrak{z}}
\newcommand{\fv}{\mathfrak{v}}
\newcommand{\fh}{\mathfrak{h}}
\newcommand{\QQ}{\mathbb{Q}}
\newcommand{\ZZ}{\mathbb{Z}}
\newcommand{\RR}{\mathbb{R}}
\newcommand{\CC}{\mathbb{C}}
\newcommand{\NN}{\mathbb{N}}
\newcommand{\FF}{\mathbb{F}}
%\newcommand{\ad}[1]{\operatorname{ad}_{#1}}

\makeatletter
\newcommand{\ad}[1]{\mathop{\operator@font ad}\nolimits_{#1}}
\makeatother

% show labels in margin (must be last package added)
%\usepackage{showlabels, yfonts}

% tiks packages
\usepackage{tikz}
\usepackage{tikzscale}
\usepackage{filecontents}
\usepackage{caption}
\usepackage{subcaption}
\usepackage{wrapfig}

\begin{filecontents*}{pic.tikz}
    \begin{tikzpicture}
        \draw (-4,4) -- (1,2);
        \draw (1, 2) -- (4,4);
        \draw (-1,6) -- (4,4);
        \draw (-1,6) -- (-4,4);
        \draw [-] (0,8) -- (0,4);
        \draw [-] (0,2.4) -- (0,0);
        %\draw [dashed] (4,0) -- (4,3);
        \node [right] at (0, 8) {$\fz$};
        \node [below] at (4,4) {$\fv$};
    \end{tikzpicture}
\end{filecontents*}

% input information for the title page here:
\preprint{}
\title{Recovering the Metric - A study of Lie Algebras}
\author{Chaskin Saroff}
\address{
   Mathematics Department\\
   State University of New York,\\
   College at Oswego\\
   Oswego NY 13126\\
   USA\\
   \textsf{csaroff@oswego.edu}
}
\date{\today} \draft
\abstract{
%You will write a short description of your work in this area. To illustrate
%how it should look, I am writing a lot of words. Your abstract should be
%concise but informative. You should only write the key ideas in the abstract.
%Do not try to explain any of your results / findings here.
We study the geometry of Lie Algebra and attempt answer the following
questions. What is a Lie Algebra? What does it mean for a Lie Algebra to have
geometry? Given an arbitrary Lie Algebra, how do we measure angles between its
elements?

We find that an arbitrary Lie Algebra's Lie Bracket can be fully represented
a sequence of Matrices, while it's inner product can be fully represented
by a single matrix.

Once a relationship was found between the Lie bracket and the inner product,
an algorithm for computing the metric was implemented.
}
\msc{}{}

\begin{document}
\maketitle

%Start with a couple of introductory paragraphs that briefly explain the
%motivation for your project and a little of the history.

\section{Preliminaries}

\begin{definition}

    A \emph{Lie Algebra} is a vector space, $\fg$, with a non-associative
    multiplication called the \emph{Lie Bracket} $[\cdot,\cdot]$ defined on it.
\\
\\The Lie Bracket is a binary operator $ [\cdot,\cdot]: \fg \times \fg \to \fg$
that satisfies the following three properties:
\\for $x,y,z \in \fg$ and $\a \in \RR$:
\begin{enumerate}
    \item  $[x,x]=\Vec{0}$,

    \item  $[x+y,z] = [x,z]+[y,z]\quad
        \text{and} \quad [\a x,y]=[x, \a y]=\a[x, y]$,

    \item  $[x,[y,z]] + [z,[x,y]] + [y,[z,x]] = \Vec{0}$.
\end{enumerate}
\end{definition}
It follows directly from (1) and (2) that $[\cdot,\cdot]$ is skew-symmetric;
\emph{i.e.} [$x,y$] = -[$y,x$] By bilinearity, every alternating product is
also skew-symmetric, regardless of the characteristic of the underlying field.
Indeed, if [ , ] is alternating then
    \begin{align}
        \Vec{0} &= [x + y, x + y]
        \\&= [x + y, x] + [x + y, y]
        \\&= [x,x] + [y, x] + [x + y, y]
        \\&= [x,x] + [y, x] + [x, y] + [y, y]
        \\\Vec{0}&=         [x, y] + [y, x]
        \\\implies [x,y] &= -[y,x]
    \end{align}
    Conversely, if $[,]$ is skew-symmetric, then
    \begin{align}
        [x,x] &= -[x,x]
        \\ [x,x]+[x,x]&=\Vec{0}
        \\ 2([x,x])&=\Vec{0}
    \end{align}
    As long as $2 \neq 0$, $[,]$ is alternating.


    In order to give the Lie Algebra geometry, we will introduce another
    operator $\langle\cdot,\cdot\rangle$ called the inner product.
\begin{definition}
    An \emph{inner product} on the vector $\fg$ is a
    \emph{real-valued, symmetric, non-degenerate, bilinear, positive definite}
    function on $\fg$.  That is, an inner product on $\fg$ is a function
    $\langle\cdot,\cdot\rangle:\fg\times\fg\to\RR$ that satifies the following
    properties:
    \\for $x,y,z \in \fg$.
    \begin{enumerate}
        \item $\langle x,y \rangle = \langle y,x \rangle$,
        \item If $\langle x,y \rangle = 0 \quad \forall y \in \fg$
            then $x = \Vec{0}$,
        \item
            $\langle x+y,z \rangle = \langle x,z \rangle + \langle y,z \rangle$
            and $\a \langle x,y \rangle = \langle \a x,y \rangle
            = \langle x,\a y \rangle$,
        \item $\langle x,x \rangle \geq 0$
            and $\langle x,x\rangle = 0 \implies x=\Vec{0}$.
    \end{enumerate}
\end{definition}
    On example of a metric Lie algebra is the vector space $\RR^3$ with the
    cross product as its Lie Bracket and the dot product as its inner product.

\begin{definition}
    The \emph{center} of a Lie Algebra, $\fg$ is
    \\\[\fz=\{z \in \fg \quad | \quad [z,x]
    = \Vec{0} \quad \forall x \in \fg\}.\]
    \\A vector, $z$, of $\fg$ is said to be in the center of $\fg$ if
    \\\[[x,z] = \Vec{0} \quad \forall x \in \fg.\]
\end{definition}

\begin{definition}
    A Lie Algebra, $\fg$ is called \emph{one-step nilpotent} if and only if
    \\\[[x,y] = \Vec{0} \quad \forall x,y \in \fg.\]
    Equivalently, $\fz = \fg.$
\end{definition}

It turns out that the center is not only a subspace of $\fg$ and therefore a
vector space, but it is also an one-step nilpotent Lie Algebra.  This is
because the Lie Bracket is defined by an identity, and the center inherits its
properties.

\begin{definition}
    A Lie Algebra, $\fg$ is called \emph{two-step nilpotent} if and only if
    \\$[[x,y],z] = \Vec{0} \quad \forall x,y,z \in \fg$
    and $\fz \neq \fg$
\end{definition}

\begin{definition}
    For any $x \in \fg$ the \emph{adjoint representation} of $\fg$ on itself is
    the function
    \\$\ad{x}(y) = [x,y] \quad \forall y \in \fg$
\end{definition}

If $\fg$ is two-step nilpotent, then for all $x \in \NN$, $\ad{x}{}$ is a
linear function
\\$\ad{x}{}: \NN \to \fz$

\begin{example}
    The Heisenberg Algebra, $\fh$,  is a specific type of Lie Algebra, that is
    spanned by three vectors $e_1, e_2, z$.
    \\It's Lie Bracket is defined on this basis by:
    \\$[e_1, e_2] = z$
    \\$[e_1, z] = \Vec{0}$
    \\$[e_2, z] = \Vec{0}$
\end{example}
The center and non-center of the Heisenberg Algebra are somewhat distinct
elements.
If $\fv$ is the vector space whose basis is $\{e_1, e_2\}$, then
$\fv$ looks like a plane, while $\fz$ is the span of $z$ and therefore looks
like a line.

\begin{figure}[h]
    \centering
    \includegraphics[width=30mm]{pic.tikz} %input file
        \label{fig:pic} %label name
    \caption{\footnotesize The Heisenberg Algebra, $\fh$} %caption
\end{figure}

\pagebreak

%Here's where you define all of the key mathematical objects that you'll
%be working with, and state ``well-known'' results such as Adams's Theorem.


\section{Results}

\section{Conclusion}

Wrap it all up in this section. You can talk about what you learned, what
you found most interesting, and directions for ``future work'' or an
alternative approach that may yield better results.

\begin{thebibliography}{99}
\bibitem{CP4}
L.\,A.\,Cordero and P.\,E.\,Parker, Pseudoriemannian 2-step
nilpotent Lie groups, \DGS\ preprint, Wichita: 2000.
{\sf arXiv:\,math/0604298}

\bibitem{K1}
A.\,Kaplan, Fundamental solutions for a class of hypoelliptic PDE
generated by composition of quadratic forms, {\it Trans. of the A.M.S.}
{\bf 258}
(1980) 147--153.

\bibitem{K2}
A.\,Kaplan, Riemannian nilmanifolds attached to Clifford modules,
{\it Geom. Dedicata} {\bf 11} (1981) 127--136.

\bibitem{K3}
A.\,Kaplan, On the geometry of groups of Heisenberg type, {\it Bull.
London Math. Soc.} {\bf 15} (1983) 35--42.

\bibitem{KT}
A.\,Kaplan and A.\,Tiraboschi, Automorphisms of Non-Singular
Nilpotent Lie Algebras, {\it J. Lie Theory} {\bf 23} (2013) 1085--1100.
\end{thebibliography}
\end{document}
