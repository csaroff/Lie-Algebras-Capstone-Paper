% template for papers with a title page
% see dgstpp.sty for title page info
% format: latex
% last changed: 1 Apr 2015

\documentclass[11 pt]{article}

% standard math packages
\usepackage{amsmath,amsfonts,amssymb}

% Phil Parker's DGS packages, some modified
\usepackage{remexpp,pprroof,dgstpp}

% other packages
\usepackage{setspace, esvect}
%\usepackage{hyperref,color}

% "fancy" font
\usepackage{fourier}
\usepackage[T1]{fontenc}
   
% make reference header the right font size
\renewcommand\refname{\Large References}
   
% theorems, remarks, etc using Phil Parker's "remexpp.sty"
\newtheorem{theorem}{Theorem}[section]
\newtheorem{prop}[theorem]{Proposition}
\newtheorem{lemma}[theorem]{Lemma}
\newtheorem{claim}[theorem]{Claim}
\newtheorem{corollary}[theorem]{Corollary}
\newremark{definition}[theorem]{Definition}
\newremark{example}[theorem]{Example}
\newremark{remark}[theorem]{Remark} 

% make rsfs, TeX \cal, and Euler script *all* available
\usepackage{mathrsfs}
\let\rscr=\mathscr
\let\mathscr=\relax
\let\mcal=\mathcal
\usepackage{eucal}
\let\escr=\mathcal
\let\mathcal=\relax

% commutative diagrams with XY-pic
\usepackage[all]{xy}
\SelectTips{cm}{}

\arraycolsep .2em
   
% new commands
\renewcommand{\a}{\alpha}
\newcommand{\Aut}[1]{\textrm{Aut}(#1)}
\newcommand{\B}{\rscr{B}}
\newcommand{\br}[2]{\left[#1,#2\right]}
\newcommand{\bre}{\br{\ }{\,}}
\newcommand{\ddg}{\ddot{\g}}
\newcommand{\dg}{\dot{\g}}
\newcommand{\DGS}{D{\kern-.375em}G{\kern-.2em}S}
\newcommand{\ds}{\oplus}
\newcommand{\eB}{\escr{B}}
\newcommand{\eH}{\escr{H}}
\newcommand{\eI}{\escr{I}}
\newcommand{\eV}{\escr{V}}
\newcommand{\g}{\gamma}
\newcommand{\G}{\Gamma}
\newcommand{\h}{\lal{h}}
\renewcommand{\H}{\rscr{H}}
\newcommand{\hp}{\h_{2p + 1}}
\newcommand{\iso}{\cong}
\newcommand{\lag}{\mathfrak{g}}
\newcommand{\lal}[1]{\mathfrak{#1}}
\newcommand{\n}{\lal{n}}
\newcommand{\pplus}{+\mspace{-10 mu}+}
\newcommand{\R}{\mathbb{R}}
\newcommand{\rS}{\rscr{S}}
%\renewcommand{\span}[1]{[\mspace{-3.25 mu}[ #1 ]\mspace{-3.25 mu}]}
\newcommand{\surj}{\rightarrow\kern-.82em\rightarrow}
\newcommand{\tQ}{\widetilde{Q}}
\renewcommand{\v}{\lal{v}}
\newcommand{\V}{\rscr{V}}
\newcommand{\z}{\lal{z}}
\newcommand{\fg}{\textfrak{g}}
\newcommand{\QQ}{\mathbb{Q}}
\newcommand{\ZZ}{\mathbb{Z}}
\newcommand{\RR}{\mathbb{R}}
\newcommand{\CC}{\mathbb{C}}
\newcommand{\NN}{\mathbb{N}}
\newcommand{\FF}{\mathbb{F}}

\makeatletter
\newcommand{\ad}[1]{\mathop{\operator@font ad}\nolimits_{#1}}
\makeatother

% show labels in margin (must be last package added)
\usepackage{showlabels, yfonts}

% input information for the title page here:
\preprint{}
\title{Recovering the Metric - A study of Lie Algebras}
\author{Chaskin Saroff}
\address{
   Mathematics Department\\
   State University of New York,\\
   College at Oswego\\
   Oswego NY 13126\\
   USA\\
   \textsf{csaroff@oswego.edu}
}
\date{\today} \draft
\abstract{
%You will write a short description of your work in this area. To illustrate
%how it should look, I am writing a lot of words. Your abstract should be
%concise but informative. You should only write the key ideas in the abstract.
%Do not try to explain any of your results / findings here.
We study the geometry of Lie Algebra and attempt answer the following questions.
What is a Lie Algebra? What does it mean for a Lie Algebra to have geometry?
Given an arbitrary Lie Algebra, how do we measure angles between its elements?

We find that an arbitrary Lie Algebra's Lie Bracket can be fully represented
a sequence of Matrices, while it's inner product can be fully represented
by one matrix.

Once a relationship was found between the Lie bracket and the inner product,
an algorithm for computing the metric was implemented.
}
\msc{}{}

\begin{document}
\maketitle

%Start with a couple of introductory paragraphs that briefly explain the
%motivation for your project and a little of the history.

\section{Preliminaries}

In order to begin exploring the geometry of a Lie Algebra, we must answer
the question, "What is a Lie Algebra?"

\begin{definition}

A Lie Algebra is a vector space with a non-associative multiplication
called the "Lie Bracket"($[\cdot,\cdot]$) defined on it.
\\
\\Let $\fg$ be a vector space over a field $\FF$ 
\\Let the binary operator $ [\cdot,\cdot]: \fg \times \fg \to \fg$ be called the "Lie Bracket" 
\\Let $x,y,z \in \fg$ and let $\a \in \FF$
\\$\fg$ is called a "Lie Algebra" if and only if it satisfies the following three properties:
\\\\(1) [x,x]=$\Vec{0}$ (Alternating)
\\\\(2) [$x+y,z$] = [$x,z$]+[$y,z$]\quad and \quad [$\a x,y$]=[$x, \a y$]=$\a$[$x, y$] (Bilinearity)
\\\\(3)  [$x$,[$y,z$]] + [$z$,[$x,y$]] + [$y$,[$z,x$]] $= \Vec{0}$ (Jacobi Identity)
\end{definition}
It follows directly that (1) and (2) that $[\cdot,\cdot]$ is anticommutative i.e. [$x,y$] = -[$y,x$]

\begin{definition}
    $<\cdot,\cdot>:\fg\times\fg\to\FF$ called the inner product.
\end{definition}


%Here's where you define all of the key mathematical objects that you'll
%be working with, and state ``well-known'' results such as Adams's Theorem.


\section{Results}
By bilinearity, every alternating product is also skew-symmetric, regardless of the characteristic of the underlying field. Indeed, if [ , ] is alternating
    then

    \begin{align}
        0 &= [x + y, x + y] 
        \\&= [x + y, x] + [x + y, y]
        \\&= [x,x] + [y, x] + [x + y, y]
        \\&= [x,x] + [y, x] + [x, y] + [y, y]
        \\0&=         [x, y] + [y, x]
        \\\implies [x,y] &= -[y,x]
    \end{align}

\section{Conclusion}

Wrap it all up in this section. You can talk about what you learned, what
you found most interesting, and directions for ``future work'' or an alternative
approach that may yield better results.

\begin{thebibliography}{99}
\bibitem{CP4}
L.\,A.\,Cordero and P.\,E.\,Parker, Pseudoriemannian 2-step
nilpotent Lie groups, \DGS\ preprint, Wichita: 2000.
{\sf arXiv:\,math/0604298}

\bibitem{K1}
A.\,Kaplan, Fundamental solutions for a class of hypoelliptic PDE
generated by composition of quadratic forms, {\it Trans. of the A.M.S.} {\bf 258}
(1980) 147--153.

\bibitem{K2}
A.\,Kaplan, Riemannian nilmanifolds attached to Clifford modules,
{\it Geom. Dedicata} {\bf 11} (1981) 127--136.

\bibitem{K3}
A.\,Kaplan, On the geometry of groups of Heisenberg type, {\it Bull.
London Math. Soc.} {\bf 15} (1983) 35--42.

\bibitem{KT} 
A.\,Kaplan and A.\,Tiraboschi, Automorphisms of Non-Singular
Nilpotent Lie Algebras, {\it J. Lie Theory} {\bf 23} (2013) 1085--1100.
\end{thebibliography}
\end{document}
