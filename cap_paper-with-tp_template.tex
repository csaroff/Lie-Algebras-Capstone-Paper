% template for papers with a title page
% see dgstpp.sty for title page info
% format: latex
% last changed: 1 Apr 2015

\documentclass[11 pt]{article}

% standard math packages
\usepackage{amsmath,amsfonts,amssymb}

% Phil Parker's DGS packages, some modified
\usepackage{remexpp,pprroof,dgstpp}

% other packages
\usepackage{setspace}
%\usepackage{hyperref,color}

% "fancy" font
\usepackage{fourier}
\usepackage[T1]{fontenc}
   
% make reference header the right font size
\renewcommand\refname{\Large References}
   
% theorems, remarks, etc using Phil Parker's "remexpp.sty"
\newtheorem{theorem}{Theorem}[section]
\newtheorem{prop}[theorem]{Proposition}
\newtheorem{lemma}[theorem]{Lemma}
\newtheorem{claim}[theorem]{Claim}
\newtheorem{corollary}[theorem]{Corollary}
\newremark{definition}[theorem]{Definition}
\newremark{example}[theorem]{Example}
\newremark{remark}[theorem]{Remark} 

% make rsfs, TeX \cal, and Euler script *all* available
\usepackage{mathrsfs}
\let\rscr=\mathscr
\let\mathscr=\relax
\let\mcal=\mathcal
\usepackage{eucal}
\let\escr=\mathcal
\let\mathcal=\relax

% commutative diagrams with XY-pic
\usepackage[all]{xy}
\SelectTips{cm}{}

\arraycolsep .2em
   
% new commands
\renewcommand{\a}{\alpha}
\newcommand{\Aut}[1]{\textrm{Aut}(#1)}
\newcommand{\B}{\rscr{B}}
\newcommand{\br}[2]{\left[#1,#2\right]}
\newcommand{\bre}{\br{\ }{\,}}
\newcommand{\ddg}{\ddot{\g}}
\newcommand{\dg}{\dot{\g}}
\newcommand{\DGS}{D{\kern-.375em}G{\kern-.2em}S}
\newcommand{\ds}{\oplus}
\newcommand{\eB}{\escr{B}}
\newcommand{\eH}{\escr{H}}
\newcommand{\eI}{\escr{I}}
\newcommand{\eV}{\escr{V}}
\newcommand{\g}{\gamma}
\newcommand{\G}{\Gamma}
\newcommand{\h}{\lal{h}}
\renewcommand{\H}{\rscr{H}}
\newcommand{\hp}{\h_{2p + 1}}
\newcommand{\iso}{\cong}
\newcommand{\lag}{\mathfrak{g}}
\newcommand{\lal}[1]{\mathfrak{#1}}
\newcommand{\n}{\lal{n}}
\newcommand{\pplus}{+\mspace{-10 mu}+}
\newcommand{\R}{\mathbb{R}}
\newcommand{\rS}{\rscr{S}}
\renewcommand{\span}[1]{[\mspace{-3.25 mu}[ #1 ]\mspace{-3.25 mu}]}
\newcommand{\surj}{\rightarrow\kern-.82em\rightarrow}
\newcommand{\tQ}{\widetilde{Q}}
\renewcommand{\v}{\lal{v}}
\newcommand{\V}{\rscr{V}}
\newcommand{\z}{\lal{z}}

\makeatletter
\newcommand{\ad}[1]{\mathop{\operator@font ad}\nolimits_{#1}}
\makeatother

% show labels in margin (must be last package added)
\usepackage{showlabels}

% input information for the title page here:
\preprint{}
\title{Recovering the Metric - A study of Lie Algebras}
\author{Chaskin Saroff}
\address{
   Mathematics Department\\
   State University of New York,\\
   College at Oswego\\
   Oswego NY 13126\\
   USA\\
   \textsf{csaroff@oswego.edu}
}
\date{\today} \draft
\abstract{
You will write a short description of your work in this area. To illustrate
how it should look, I am writing a lot of words. Your abstract should be
concise but informative. You should only write the key ideas in the abstract.
Do not try to explain any of your results / findings here.
}
\msc{}{}

\begin{document}
\maketitle


Start with a couple of introductory paragraphs that briefly explain the
motivation for your project and a little of the history.



\section{Preliminaries}

Here's where you define all of the key mathematical objects that you'll
be working with, and state ``well-known'' results such as Adams's Theorem.


\section{Results}

In this section you will describe your work, including some of your code (if
you want) and results from Sage.

\section{Conclusion}

Wrap it all up in this section. You can talk about what you learned, what
you found most interesting, and directions for ``future work'' or an alternative
approach that may yield better results.

\begin{thebibliography}{99}
\bibitem{CP4}
L.\,A.\,Cordero and P.\,E.\,Parker, Pseudoriemannian 2-step
nilpotent Lie groups, \DGS\ preprint, Wichita: 2000.
{\sf arXiv:\,math/0604298}

\bibitem{K1}
A.\,Kaplan, Fundamental solutions for a class of hypoelliptic PDE
generated by composition of quadratic forms, {\it Trans. of the A.M.S.} {\bf 258}
(1980) 147--153.

\bibitem{K2}
A.\,Kaplan, Riemannian nilmanifolds attached to Clifford modules,
{\it Geom. Dedicata} {\bf 11} (1981) 127--136.

\bibitem{K3}
A.\,Kaplan, On the geometry of groups of Heisenberg type, {\it Bull.
London Math. Soc.} {\bf 15} (1983) 35--42.

\bibitem{KT} 
A.\,Kaplan and A.\,Tiraboschi, Automorphisms of Non-Singular
Nilpotent Lie Algebras, {\it J. Lie Theory} {\bf 23} (2013) 1085--1100.
\end{thebibliography}
\end{document}
